\chapter{内存SSA算法伪代码}

本附录为内存SSA转换步骤的算法伪代码,算法说明请参考第\ref{sec:ssa}节。

\begin{figure}[htbp]
\centering
\begin{tabular}{c}
\begin{lstlisting}
// Step1. place all necessary PHINodes for memory objects
for(each BasicBlock bb in F) {

  for(each Instruction inst in bb) {
    if(inst is a StoreInst) {
      // If a StoreInst modify a memory object aliasing our 
      // 'resources', insert a PHINode at the dominance frontier.
      Value* ptr = store->getPointerOperand();

      // Get the pointed variable equivalent class (a DSNode )
      //   from the DSA analysis result
      DSNode *n = dsgraph->getNodeForValue(ptr).getNode(); 

      for(each DominanceFrontier frontier of bb) {
          Create PhiNode for n at frontier;
      }

    } else if(inst is an AllocInst) {
      // AllocaInst is treated as storing an 'unspecific'
      //   value to the memory.
      Value *ptr = alloca;
      DSNode *n = dsgraph->getNodeForValue(ptr).getNode();

      for(each DominanceFrontier frontier of bb) {
          Create PhiNode for n at frontier;
      }
    } else if(inst is a CallInst){
      ...
    }
  }
}
\end{lstlisting}
\end{tabular}
\caption{$\phi$放置第一步:对每一条修改变量的指令放置一个$\phi$指令在支配边界处。}\label{alg:phi1}
\end{figure}

\begin{figure}[htbp]
\centering
\begin{tabular}{c}
\begin{lstlisting}
// Add PHINode for PHINode inserted earlier 
//   until the PHINode set is closed...
queue<BasicBlock*> bbWithNewPHI;
for(each BasicBlock 'bb' with a PHINode) {
  bbWithNewPHI.push(bb);
}
while(!bbWithNewPHI.empty()) {
  BasicBlock* bb = bbWithNewPHI.front();
  bbWithNewPHI.pop();

  for(each PHINode p of bb) {
    DSNode *n = Coresponding DSNode of p;
    for(each DomianceFrontier frontier of bb) {
      if(frontier does not have a PHINode for n) {
        Create PhiNode for n at frontier;
        if(frontier is not in bbWithPHI) {
          bbWithNewPHI.push(frontier);
        }
      }
    }
  }
}
\end{lstlisting}
\end{tabular}
\caption{$\phi$放置第二步:求$\phi$结点的闭包}\label{alg:phi2}
\end{figure}


\begin{figure}[htbp]
\centering
\begin{tabular}{c}
\begin{lstlisting}

void buildSSARenaming(map<DSNode *, Instruction *> &lastDef,
     BasicBlock *bb) {
  // Step1. scan instructions in 'bb', adding new definitions 
  //   and linking usage to its last definition.
  lastDefBackup = lastDef;
  for(each PHINode p of bb) {
    DSNode *n = corresponding variable equivalent class of p
    lastDef[n] = p;
  }
  for(each Instruction inst in bb) {
    // Alloca/StoreInst use a value and define a new version of it.
    // LoadInst only use a value.
    if(inst is a StoreInst) {
      Value *ptr = inst->getPointerOperand();
      DSNode *n = dsgraph->getNodeForValue(ptr).getNode();
      Instruction *def = lastDef[n];
      memSSAUsers[def].insert(store);
      // Defines a new
      lastDef[n] = inst;
    } else if(inst is a LoadInst) {
      Value *ptr = inst->getPointerOperand();
      DSNode *n = dsgraph->getNodeForValue(ptr).getNode();
      // Reads an old
      Instruction *def = lastDef[n];
      memSSAUsers[def].insert(inst);
    } else { ... /* deal with other kinds of instructions */}
  // Step2. update PHINodes for successors blocks in the CFG.
  for(BasicBlock *succ : successors(bb)) {
      for(each PHINode p of succ) {
        DSNode *n = corresponding DSNode;
        if(lastDef[n] exists) {
          Instruction *def = lastDef[n];
          memSSAUsers[def].insert(p);
        }
      }
  }
  // Step3. Recursively process immediate dominated BasicBlocks.
  for(each child of bb in DominanceTree)) {
    buildSSARenaming(lastDef, child->getBlock());
  }
  // Step4. Restore definitions added in this BasicBlock.
  lastDef = lastDefBackup;
}
\end{lstlisting}
\end{tabular}
\caption{$\phi$放置第三步:解析定义使用关系}\label{alg:phi3}
\end{figure}