% Copyright (c) 2014,2016 Casper Ti. Vector
% Public domain.

\begin{cabstract}
指针分析是编译优化、程序分析领域重要的基础问题之一。因指针分析问题在可计算性上是不可判定的,所以对该问题的研究的主要角度是算法的“精确度”与算法的“运行效率”之间的平衡。

近四十年来,学界已有大量工作设计不同准确率、不同计算效率的指针分析算法,近年来较有影响力的包括Lattner等人的上下文敏感分析算法DSA\supercite{Lattner2007},及Hardekopf等人的流敏感稀疏分析算法\supercite{hardekopf2011flow}。本文对这两类算法在数据流分析框架下进行一致的理论分析,并在同一理论框架下设计出一个同时流敏感、上下文敏感的指针分析算法。

实验表明本文提出的算法在精度上相较于学界以往成果皆有大幅度改善并同时拥有在实践中可接受的运行效率。在作者所知范围内,这是学界中报告的首个高效流敏感、上下文敏感指针分析算法。

本文同时提出一个用我们的指针分析算法进行内存泄漏自动修复的方法。实验结果表明,我们的算法可以大幅改进学界在此问题上的最先进成果。
\end{cabstract}

\begin{eabstract}
	Test of the English abstract.
\end{eabstract}

% vim:ts=4:sw=4
