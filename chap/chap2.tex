\chapter{上下文、流、堆敏感指针分析算法CFHS}\label{algorithm}

本章介绍我们设计的上下文敏感、流敏感、堆上下文敏感基于合一的指针分析算法,简便起见,我们称之为CFHS算法( Context-Flow-Heap-Sensitive Algorithm)。CFHS算法是一个分阶段的算法,其大致流程为:

\begin{enumerate}[topsep=1em, itemsep=1em]
\item 其首先使用其他任意指针分析(称之为“辅助分析”)对程序进行快速而粗略的指针分析。
\item 根据辅助分析的结果将程序转换为完全静态单赋值形式,得到“定义-使用图” ( Def-Use Graph, DUG)。
\item 根据辅助分析的结果,识别出函数调用图。并且对函数调用图中的每一个强连通分量(Strong Connected Component, SCC ),将每个函数的DUG拼接合并为SCC级别的超级DUG。
\item 对于每个SCC的DUG,进行稀疏流敏感指针分析,并对每个函数生成“归纳”。
\item 在SCC的无环调用图中,后序地处理每个SCC,对每个调用指令,将被调用函数的归纳应用到调用指令处,并更新SCC的流敏感分析结果及摘要。
\end{enumerate}

在本章接下来的各节中,我们详细说明算法的每一个步骤,并给出伪代码及实例说明。

\section{内存SSA变换}\label{sec:ssa}
所谓静态单赋值形式(SSA)\supercite{Cytron:1991}指的是程序中每一个变量只被定义恰好一次。变量可以被多次定义的原始形式的程序被转换为SSA形式后,被多次定义的原始变量被分割为不同的实例,每个实例对应一次定义。在程序控制流图的交汇点,同一个变量的不同实例通过$ \phi $函数进行合并,产生一个这个变量的一个新的实例。SSA形式非常适于进行稀疏数据流分析,因为它显式地表示出了“定义-使用”关系,使得数据流信息可以直接沿着变量定义-变量使用边进行传播\supercite{Reif:1977}。图\ref{code:ssa}是一个原始形式程序与相应的SSA形式的例子。

\begin{figure}[htbp]
\centering
\begin{subfigure}{0.5\linewidth}
\centering
\begin{BVerbatim}
a = 0;
if(...) {
    a = 1;
} else {
    a = 2;
}
\end{BVerbatim}
\caption{原始形式}
\end{subfigure}%
\begin{subfigure}{0.5\linewidth}
\centering
\begin{tabular}{c}
\begin{lstlisting}
a$_0$ = 0;
if(...) {
  a$_1$ = 1;
} else {
  a$_2$ = 1;
}
a$_3$ = $\phi$(a$_1$, a$_2$);
\end{lstlisting}
\end{tabular}
\caption{SSA形式}
\end{subfigure}
\caption{SSA形式示例}\label{code:ssa}
\end{figure}

\subsection{辅助指针分析}

指针的存在使得SSA转换变得困难,所以现代编译器的中间表示格式,比如LLVM IR \supercite{llvm},往往是“部分SSA形式”,即被取过地址的变量仍然是可修改的内存中的变量形式,而没有被取过地址的变量则可以被优化为静态单赋值形式。而如果我们想把程序中的所有变量转换为SSA形式,则必须要进行指针分析,从而知道通过指针进行的间接访问修改的是哪些变量。但因为指针分析生成的是“保守”的结果(即指针分析结果认为 p 指向 a ,实际上 p 不一定真的会指向 a ,但指针分析一定包含真实的结果),所以通过指针分析识别出的间接访问、修改实际只是“潜在“间接修改。我们也采用 Chow等人\supercite{Chow:1996}的方法,对于间接写入(如 \verb|*x = y| ) 我们会把其视作对  \verb|x| 所有可能指向的 \verb|a| 进行既读取又修改,对于间接读取(如 \verb|y = *x|),我们把其当作当作对 \verb|x| 所有可能指向的 \verb|a| 进行了读取。

辅助分析的选择只要是安全的即可,Hardekopf等人的研究工作\supercite{Hardekopf2011}中选择的是流不敏感、上下文不敏感的基于子集的分析算法( Andersen算法 )。而在我们的研究工作中,我们选择使用Lattner等人的DSA算法。DSA算法相对于Andersen算法效率更高,且因为DSA算法是基于合一的,所以在分析结果中,每个指针总是指向最多一个“变量等价类”,所以我们在构造SSA形式时,可以以变量等价类为基本单位,从而简化算法并在某些情况下改进算法的效率。而如果使用 Andersen算法,则需要像SFS算法一样进行额外的识别变量等价类的步骤(\cite{Hardekopf2011}中的 Access Equivalence 优化)。

% 中间形式示例
我们通过图\ref{code:ssa2}中的程序对我们的转换方法进行示例。在图\ref{code:ssa2:l}的程序中,指针 \verb|p| 既可能指向 \verb|a| 也可能指向 \verb|b| ,所以对于流不敏感、基于合一的DSA会将 \verb|a| 和 \verb|b| 视为一个“变量等价类”,并且 \verb|q| 也指向这一个等价类。根据辅助指针分析提供的信息,我们可以知道原始程序每条语句都直接或间接地修改哪些变量。图\ref{code:ssa3}显式地在程序中标注出了通过指针进行间接访问的变量。按照Chow等人的记号\supercite{Chow:1996}, 我们对间接读取用 $\mu()$ 、间接写入用 $\chi()$ 这两个虚拟函数表示。

\begin{figure}
\centering
\begin{subfigure}{0.5\linewidth}
\centering
\begin{tabular}{c}
\begin{lstlisting}
int a, b;
int *p, *q;
if(...) {
    p = &b;
    q = &a;
    *p = a;
    b = *q;
} else {
    p = &a;
    *p = 0;
}
a = b;
\end{lstlisting}
\end{tabular}
\caption{}\label{code:ssa2:l}
\end{subfigure}%
\begin{subfigure}{0.5\linewidth}
\centering
\centering
\begin{tikzpicture}[node distance=1cm and 1cm]
\node[round node] (p) {p};
\node[right=of p] (c) {};
\node[round node, right=of c] (q) {q};
\node[round node, below=of c] (ab) {a, b};
\graph{
    (p) -> (ab);
    (q) -> (ab);    
};
\end{tikzpicture}
\caption{}    
\end{subfigure}
\caption{左图:一段示例程序。右图:DSA生成的这段程序的指向图。}
\label{code:ssa2}
\end{figure}

\begin{figure}
\centering
\begin{tabular}{c}
\begin{lstlisting}
int a, b;
int *p, *q;
if(...) {
    p = &b;
    q = &a;
    *p = a; ab = $\chi(\text{ab})$; 
    b = *q; $\mu(\text{ab})$;
} else {
    p = &a;
    *p = 0; ab = $\chi(\text{ab})$;
}
a = b;
\end{lstlisting}
\end{tabular}
\caption{显式标注了通过间接读取修改的变量的程序}\label{code:ssa3}
\end{figure}

% SSA放置算法
\subsection{$\phi()$函数放置}
在知道每条语句可能会修改哪些变量后,我们可以使用标准算法\supercite{aycock2000simple}\supercite{bilardi2003algorithms}\supercite{cytron1995efficiently}\supercite{Cytron:1991}将代码转换为静态单赋值形式。我们实现的算法按如下几个步骤进行:

\begin{enumerate}[topsep=1em, itemsep=1em]
\item 对每一条修改变量的指令放置一个$\phi$指令在支配边界处。在我们的实现中,我们直接使用 LLVM 提供的支配树分析结果来获取每个基本块的支配边界。算法伪代码见图\ref{alg:phi1}。注意我们对分配变量的指令 \verb|AllocInst| 理解为它也修改了它分配的变量的值为“任意值”。
\item 然后我们对所有新添加的 $\phi$ 节点也在它们的支配边界放置新的$\phi$结点,直到所有的$\phi$结点都被这样处理。如图\ref{alg:phi2}。
\item 遍历程序,对每条访问内存变量的指令解析出它访问的变量的上一次“定义”在哪里。方法是:按直接支配关系遍历每个基本块,同时维护每个变量的最后一次定义的位置。对于每个基本块,首先对基本块内每条指令根据维护的“最后一次定义位置”解析出它们的“定义-使用边”,然后对这个基本块在CFG中的所有后继基本块,更新后继基本块开头的$\phi$结点的参数,最后递归处理基本块在支配树上的儿子基本块(既被本基本块直接支配的基本块)。算法如图\ref{alg:phi3}。
\end{enumerate}

对于图\ref{code:ssa3}中的程序,经过SSA放置算法处理之后的结果可参见图\ref{}。

% 过程间memSSA
\subsection{过程间内存SSA}

\section{局部分析阶段}

\section{自下而上分析阶段}
% vim:ts=4:sw=4